% -*- coding: utf-8; mode: latex; mode: visual-line; mode: flyspell -*-
\documentclass[twocolumn]{article}

\usepackage{etex}

\usepackage[utf8]{inputenc}
\usepackage[T1]{fontenc}
\usepackage[english]{babel}
\usepackage[babel=true,activate={true,nocompatibility},final,tracking=true,kerning=true,spacing=true]{microtype}
\usepackage{lmodern}

\usepackage{amsmath}
\usepackage{amssymb}
\usepackage{bcprules}
\usepackage{multicol}
\usepackage{stmaryrd}
\usepackage{xspace}

\newcommand{\bnfdef}{::=}
\newcommand{\bnfalt}{\mathrel{\mid}}
\newcommand{\kw}[1]{\mbox{#1}}
\newcommand{\arr}{\mathrel{<\!\!<\!\!<}}
\newcommand{\parr}{\mathrel{|\!\!<\!\!<\!\!<\!\!|}}
\newcommand{\types}[3]{#1 \vdash #2 \mathrel{:} #3}
\newcommand{\join}[2]{#1 \sqcup #2}
\newcommand{\clos}[2]{\left<#1; #2\right>}

\newcommand{\thread}[4]{\left<#1; #2; #3; #4\right>}
\newcommand{\proc}[3]{\ll\!#1; #2; #3\!\gg}
\newcommand{\mach}[2]{\parallel\!#1; #2\!\parallel}

\newcommand{\queue}[1]{\left<#1\right>}
\newcommand{\tctx}[5]{\thread{#1}{#2}{#3}{#4}\left[#5\right]}
\newcommand{\mctx}[3]{\mach{#1}{#2}\left[#3\right]}

\newcommand{\ox}[1]{\llbracket#1\rrbracket}
\newcommand{\tstep}{\rightarrow}
\newcommand{\mstep}{\Rightarrow}
\newcommand{\rref}[1]{\textsc{#1}}

\newcommand{\figurebox}[1]{\fbox{\begin{minipage}{\textwidth} #1 \medskip\end{minipage}}}

\newcommand{\judgment}[2]{{\bf #1} \hfill #2}

\typicallabel{P-ParSpawn}
\bcprulestwocoltrue
%%\bcprulessavespacetrue

\title{Ziria Operational Semantics}
\author{Geoffrey Mainland}

\begin{document}
\maketitle

\section{Language and Typing Relation}

The semantics given in this document are implemented in PLT Redex in the file \texttt{ziria.rkt}.

The expression language in Figure~\ref{fig:lang:exp} is essentially identical to the
existing Ziria language, but all syntactic categories have been collapsed into
the single category of expressions. In exchange, the type language in
Figure~\ref{fig:lang:type} is rather complicated.

Base types, $\tau$, are as expected. Computation types, represented by $\kappa$,
are of the form $\kw{ST}~ \iota~ \alpha~ \beta$. Function arguments, of type
$\rho$, may be of type $\tau$ or $\kw{ref}~ \tau$. Function results, $\sigma$,
may be of type $\tau$ or $\kappa$; that is, functions may return
computations. The typing context may contain values of type $\rho$ (base types
and refs) or of type $\rho_1 \ldots \rho_n \to \sigma$ (functions). The typing
judgment assigns expressions a type $\theta$, which may be a base type, ref
type, computation, or function.

\begin{figure}
\begin{equation*}
\begin{array}{lcl}
  e, c
     & \bnfdef & k \\
     & \bnfalt & x \\
     & \bnfalt & unop~ e \\
     & \bnfalt & e~ binop~ e \\
     & \bnfalt & \kw{if}~ e~ \kw{then}~ e~ \kw{else}~ e \\
     & \bnfalt & \kw{let}~ x \mathrel{=} e~ \kw{in}~ e \\
     & \bnfalt & \kw{letfun}~ f~ (x_1 : \rho_1, \ldots, x_n : \rho_n) \mathrel{=} e~ \kw{in}~ e \\
     & \bnfalt & \kw{letref}~ x \mathrel{=} e~ \kw{in}~ e \\
     & \bnfalt & f~ e_1~ \ldots~ e_n \\
     & \bnfalt & \kw{!} x \\
     & \bnfalt & x \mathrel{:=} e \\
     & \bnfalt & \kw{return}~ e \\
     & \bnfalt & x \mathrel{\leftarrow} c;~ c \\
     & \bnfalt & c;~ c \\
     & \bnfalt & \kw{take} \\
     & \bnfalt & \kw{emit}~ e \\
     & \bnfalt & \kw{repeat}~ c \\
     & \bnfalt & c \arr c \\
     & \bnfalt & c \parr c \\
  k
     & \bnfdef & () \\
     & \bnfalt & \kw{true} \\
     & \bnfalt & \kw{false} \\
     & \bnfalt & n \\
  unop
     & \bnfdef & - \\
  binop
     & \bnfdef & \ne \\
     & \bnfalt & = \\
     & \bnfalt & + \\
     & \bnfalt & - \\
     & \bnfalt & \times \\
     & \bnfalt & \kw{div}
\end{array}
\end{equation*}
\caption{Expression language}
\label{fig:lang:exp}
\end{figure}

\begin{figure}
\begin{equation*}
\begin{array}{lcl}
  \alpha, \beta, \gamma, \nu, \tau
     & \bnfdef & () \\
     & \bnfalt & \kw{bool} \\
     & \bnfalt & \kw{int} \\
  \iota
     & \bnfdef & \kw{C}~ \tau \\
     & \bnfalt & \kw{T} \\
  \kappa
     & \bnfdef & \kw{ST}~ \iota~ \tau~ \tau~ \\
  \rho
     & \bnfdef & \tau \\
     & \bnfalt & \kw{ref}~ \tau \\
  \sigma
     & \bnfdef & \tau \\
     & \bnfalt & \kappa \\
  \phi
     & \bnfdef & \rho \\
     & \bnfalt & \rho_1 \ldots \rho_n \to \sigma \\
  \Gamma
     & \bnfdef & \cdot \\
     & \bnfalt & x : \phi, \Gamma \\
  \theta
     & \bnfdef & \tau \\
     & \bnfalt & \kw{ref}~ \tau \\
     & \bnfalt & \kappa \\
     & \bnfalt & \rho_1 \ldots \rho_n \to \sigma
\end{array}
\end{equation*}
\caption{Type language}
\label{fig:lang:type}
\end{figure}
The base typing rules are given in Figure~\ref{type:base}. The rules for
computations are given in Figure~\ref{type:comp}. Both monadic sequencing and
monadic bind are including because treating sequencing as a derived form was
slightly more complicated in Redex---it was easier not to generate a fresh
variable for the binding. Notice that $\kw{map}$ has been removed. The treatment
of $\arr$ and $\parr$ is also slightly different from the old formulation;
$\arr$ is \emph{sequential} transformer composition, and $\parr$ is
\emph{parallelizable} transformer composition. We may want all transformer
composition to be parallelizable in the surface language, where $\parr$ serves
only as an annotation indicating to the compiler that it should run the compose
transformers in separate threads, but I've kept the distinction in the core
language. Note that we only split the typing context for $\parr$; the intent is
that serial transformer composition spawns two threads that operate in lockstep,
thus allowing for a shared heap.

\begin{figure*}
\figurebox{
\begin{multicols}{2}[\judgment{Typing Relation}{\fbox{$\types{\Gamma}{e}{\theta}$}}]
\infax[T-Unit]
  {\types{}{()}{()}}

\infax[T-True]
  {\types{}{\kw{true}}{\kw{bool}}}

\infax[T-False]
  {\types{}{\kw{false}}{\kw{bool}}}

\infax[T-Int]
  {\types{}{n}{\kw{bool}}}

\infrule[T-Var]
  {x:\phi \in \Gamma}
  {\types{\Gamma}{x}{\phi}}

\infrule[T-Unop]
  {\types{\Gamma}{e}{\phi_1}}
  {\types{\Gamma}{unop~ e}{unop_{\phi_1}(unop)}}

\infrule[T-Binop]
  {\types{\Gamma}{e_1}{\phi_1} \andalso \types{\Gamma}{e_2}{\phi_2}}
  {\types{\Gamma}{e_1~ binop~ e_2}{binop_{\phi_1, \phi_2}(binop)}}

\infrule[T-If]
  {\types{\Gamma}{e_1}{\kw{bool}} \andalso
    \types{\Gamma}{e_2}{\phi} \andalso
    \types{\Gamma}{e_3}{\phi}}
  {\types{\Gamma}{\kw{if}~ e_1~ \kw{then}~ e_2~ \kw{else}~ e_3}{\phi}}

\infrule[T-Let]
  {\types{\Gamma}{e_1}{\tau_1} \andalso
    \types{x : \tau_1, \Gamma}{e_2}{\phi}}
  {\types{\Gamma}{\kw{let}~ x \mathrel{=} e_1~ \kw{in}~ e_2}{\phi}}

\infrule[T-LetFun]
  {\types{x_1 : \rho_1, \ldots, x_n : \rho_n, \Gamma}{e_1}{\sigma} \\
    \types{f : \rho_1 \ldots \rho_n \to \sigma, \Gamma}{e_2}{\theta}}
  {\types{\Gamma}{\kw{letfun}~ f~ (x_1 : \rho_1, \ldots, x_n : \rho_n) \mathrel{=} e_1~ \kw{in}~ e_2}{\phi}}

\infrule[T-Call]
  {\types{\Gamma}{f}{\rho_1 \ldots \rho_n \to \sigma} \\
    \types{\Gamma}{e_1}{\rho_1} \andalso \ldots \andalso \types{\Gamma}{e_n}{\rho_n}}
  {\types{\Gamma}{f~ e_1~ \ldots e_n}{\sigma}}
\end{multicols}
}
\caption{Base typing rules}
\label{type:base}
\end{figure*}

\begin{figure*}
\figurebox{
\begin{multicols}{2}[\judgment{Typing Relation}{\fbox{$\types{\Gamma}{e}{\theta}$}}]
\infrule[T-LetRef]
  {\types{\Gamma}{e_1}{\tau_1} \andalso
    \types{x : \kw{ref}~ \tau_1, \Gamma}{e_2}{\kw{ST}~ \iota~ \alpha~ \beta}}
  {\types{\Gamma}{\kw{letref}~ x \mathrel{=} e_1~ \kw{in}~ e_2}{\kw{ST}~ \iota~ \alpha~ \beta}}

\infrule[T-Deref]
  {x:\kw{ref}~ \tau \in \Gamma}
  {\types{\Gamma}{!x}{\kw{ST}~ (\kw{C}~ \tau)~ \alpha~ \beta}}

\infrule[T-Set]
  {x:\kw{ref}~ \tau \in \Gamma \andalso \types{\Gamma}{e}{\tau}}
  {\types{\Gamma}{x \mathrel{:=} e}{\kw{ST}~ (\kw{C}~ ())~ \alpha~ \beta}}

\infrule[T-Return]
  {\types{\Gamma}{c}{\tau}}
  {\types{\Gamma}{\kw{return}~ c}{\kw{ST}~ (\kw{C}~ \tau)~ \alpha~ \beta}}

\infrule[T-Bind]
  {\types{\Gamma}{c_1}{\kw{ST}~ (\kw{C}~ \nu)~ \alpha~ \beta} \andalso
   \types{x : \nu, \Gamma}{c_2}{\kw{ST}~ \iota~ \alpha~ \beta}}
  {\types{\Gamma}{x \mathrel{\leftarrow} c_1;~ c_2}{\kw{ST}~ \iota~ \alpha~ \beta}}

\infrule[T-Seq]
  {\types{\Gamma}{c_1}{\kw{ST}~ (\kw{C}~ \nu)~ \alpha~ \beta} \andalso
   \types{x\Gamma}{c_2}{\kw{ST}~ \iota~ \alpha~ \beta}}
  {\types{\Gamma}{c_1;~ c_2}{\kw{ST}~ \iota~ \alpha~ \beta}}

\infax[T-Take]
  {\types{\Gamma}{\kw{take}}{\kw{ST}~ (\kw{C}~ \alpha)~ \alpha~ \beta}}

\infrule[T-Emit]
  {\types{\Gamma}{e}{\beta}}
  {\types{\Gamma}{\kw{emit}~ e}{\kw{ST}~ (\kw{C}~ ())~ \alpha~ \beta}}

\infrule[T-Repeat]
  {\types{\Gamma}{c}{\kw{ST}~ (\kw{C}~ ())~ \alpha~ \beta}}
  {\types{\Gamma}{\kw{repeat}~ c}{\kw{ST}~ \kw{T}~ \alpha~ \beta}}

\infrule[T-Arr]
  {\types{\Gamma}{c_1}{\kw{ST}~ \iota_1~ \alpha~ \beta} \\
   \types{\Gamma}{c_2}{\kw{ST}~ \iota_2~ \alpha~ \beta}}
  {\types{\Gamma}{c_1 \arr c_2}{\kw{ST}~ (\join{\iota_1}{\iota_2})~ \alpha~ \beta}}

\infrule[T-PArr]
  {\Gamma_1 \oplus \Gamma_2 = \Gamma \\
   \types{\Gamma_1}{c_1}{\kw{ST}~ \iota_1~ \alpha~ \beta} \\
   \types{\Gamma_2}{c_2}{\kw{ST}~ \iota_2~ \alpha~ \beta}}
  {\types{\Gamma}{c_1 \parr c_2}{\kw{ST}~ (\join{\iota_1}{\iota_2})~ \alpha~ \beta}}
\end{multicols}
}
\caption{Computation typing rules}
\label{type:comp}
\end{figure*}

\begin{figure}
\begin{align*}
\join{\kw{C}~ \nu}{\kw{T}} &= \kw{C}~ \nu \\
\join{\kw{T}}{\kw{C}~ \nu} &= \kw{C}~ \nu \\
\join{\kw{T}}{\kw{T}}      &= \kw{T}
\end{align*}
\caption{Definition of $\join{\cdot}{\cdot}$}
\label{def:join}
\end{figure}

\section{Evaluation}

\begin{figure}
\begin{equation*}
\begin{array}{lcl}
  E
     & \bnfdef & unop~ E \\
     & \bnfalt & E~ binop~ e \\
     & \bnfalt & v~ binop~ E \\
     & \bnfalt & \kw{if}~ E~ \kw{then}~ e~ \kw{else}~ e \\
     & \bnfalt & \kw{let}~ x \mathrel{=} E~ \kw{in}~ e \\
     & \bnfalt & f~ v_1~ \ldots~ v_{i-1}~ E~ e_{i+1}~ \ldots~ e_n \\
     & \bnfalt & \kw{letref}~ x \mathrel{=} E~ \kw{in}~ e \\
     & \bnfalt & x \mathrel{:=} E \\
     & \bnfalt & \kw{return}~ E \\
     & \bnfalt & x \mathrel{\leftarrow} E;~ e \\
     & \bnfalt & E;~ e \\
     & \bnfalt & \kw{emit}~ E \\
     & \bnfalt & \cdot \\
  v
     & \bnfdef & k \\
  clos
     & \bnfdef & \clos{x_1, \ldots, x_n}{e} \\
  v_\sigma
     & \bnfdef & v \\
     & \bnfalt & \ell \\
     & \bnfalt & clos \\
  \Sigma
     & \bnfdef & \emptyset \\
     & \bnfalt & x = v_\sigma, \Sigma \\
  H
     & \bnfdef & \emptyset \\
     & \bnfalt & \ell = v, H \\
  \delta
     & \bnfdef & \kw{tick} \\
     & \bnfalt & \kw{wait} \\
     & \bnfalt & \kw{cons}~ v \\
     & \bnfalt & \kw{yield}~ v \\
  t
     & \bnfdef & \thread{\Sigma}{H}{e}{\delta} \\
  Q
     & \bnfdef & \kw{queue}~ v \ldots \\
     & \bnfalt & \kw{mvar} \\
     & \bnfalt & \kw{mvar}~ v \\
  p
     & \bnfdef & \proc{\thread{\Sigma}{H}{e}{\delta}}{q_i}{q_o} \\
  \Phi
     & \bnfdef & \emptyset \\
     & \bnfalt & q = Q, \Phi \\
\end{array}
\end{equation*}
\caption{Evaluation contexts}
\label{fig:lang:context}
\end{figure}

Evaluation contexts are given in Figure~\ref{fig:lang:context}. Closures are
represented by $clos$. Note that the semantics for function call is not
correct---do we want proper closures or not?

The store, $\Sigma$, maps variables to values, locations ($\ell$), or
closures. The heap, $H$, maps locations to values.

Threads consist of a store, a heap, an expression, and an \emph{action},
$\delta$. This action represents the thread's interaction with the outside world
via queues; note that threads \emph{can} share heaps, but only if the operate in
lockstep. Otherwise all communication between threads is via queues.

Processes consist of a thread and two queues, an input queue and an output
queue.

\emph{Machines} consist of a ``queue heap,'' $\Phi$, that maps queue locations
$q$ to queues, and a collection of processes. Queues may be proper queues, which
can contain zero or more values, denoted by $\kw{queue}$, or message boxes,
which may contain zero or one value, denoted by $\kw{mvar}$. Sequential
transformer composition synchronizes on message boxes, and parallel transformer
composition synchronizes on proper queues.

Evaluation is specified in the form of two small-step relations, one for
threads, and one for machines. Figure~\ref{fig:eval:base} and
Figure~\ref{fig:eval:comp} give the thread evaluation relation. The $\kw{take}$
and $\kw{emit}$ expressions cause the thread to synchronize via the $\delta$
action; $\kw{take}$ waits for a value, and $\kw{emit}$ yields a value. The nice
thing about this alternative (with respect to the semantics from the NSDI
submission) formulation is that is is small-step, allowing us to reason about
threads running in parallel, and the domain and codomain are identical.

The evaluation relation for machines is given in Figure~\ref{fig:eval:mach}. If
one thread can step, then the machine can step. If a thread needs to synchronize
via its input/output queues, then the machine can step by either enqueuing or
dequeuing values to satisfy a thread's action, $\delta$.

\begin{figure*}
\figurebox{
\judgment{Thread Evaluation}{\fbox{$\thread{\Sigma_1}{H_1}{e_1}{\delta_1} \tstep \thread{\Sigma_2}{H_2}{e_2}{\delta_2}$}}
\infax[E-Var]
  {\tctx{\Sigma}{H}{E}{\kw{tick}}{x} \tstep
   \tctx{\Sigma}{H}{E}{\kw{tick}}{\Sigma\left[x\right]}}

\infax[E-Unop]
  {\tctx{\Sigma}{H}{E}{\kw{tick}}{unop~ n} \tstep
   \tctx{\Sigma}{H}{E}{\kw{tick}}{\ox{unop~n}}}

\infax[E-Binop]
  {\tctx{\Sigma}{H}{E}{\kw{tick}}{n_1~ binop~ n_2} \tstep
   \tctx{\Sigma}{H}{E}{\kw{tick}}{\ox{n_1~ binop~ n_2}}}

\infax[E-IfTrue]
  {\tctx{\Sigma}{H}{E}{\kw{tick}}{\kw{if}~ \kw{true}~ \kw{then}~ e_1~ \kw{else}~ e_2} \tstep
   \tctx{\Sigma}{H}{E}{\kw{tick}}{e_1}}

\infax[E-IfFalse]
  {\tctx{\Sigma}{H}{E}{\kw{tick}}{\kw{if}~ \kw{false}~ \kw{then}~ e_1~ \kw{else}~ e_2} \tstep
   \tctx{\Sigma}{H}{E}{\kw{tick}}{e_2}}

\infax[E-Let]
  {\tctx{\Sigma}{H}{E}{\kw{tick}}{\kw{let}~ x \mathrel{=} v~ \kw{in}~ e} \tstep
   \tctx{\Sigma\left[x \mapsto v\right]}{H}{E}{\kw{tick}}{e}}

\infax[E-LetFun]
  {\tctx{\Sigma}{H}{E}{\kw{tick}}{\kw{letfun}~ f~ (x_1 : \rho_1, \ldots, x_n : \rho_n) \mathrel{=} e_1~ \kw{in}~ e_2} \tstep \\
   \tctx{\Sigma\left[f \mapsto \clos{x_1, \ldots, x_n}{e_1}\right]}{H}{E}{\kw{tick}}{e_2}}

\infax[E-Call]
  {\tctx{\Sigma}{H}{E}{\kw{tick}}{f~ v_1~ \ldots v_n} \tstep \\
   \tctx{\Sigma\left[x_1 \mapsto v_1, \ldots, x_n \mapsto v_n\right]}{H}{E}{\kw{tick}}{e} \\
   \textrm{where $\Sigma[f] = \clos{x_1, \ldots, x_n}{e}$}}
}
\caption{Base thread evaluation relation}
\label{fig:eval:base}
\end{figure*}

\begin{figure*}
\figurebox{
\judgment{Thread Evaluation}{\fbox{$\thread{\Sigma_1}{H_1}{e_1}{\delta_1} \tstep \thread{\Sigma_2}{H_2}{e_2}{\delta_2}$}}
\infax[E-LetRef]
  {\tctx{\Sigma}{H}{E}{\kw{tick}}{\kw{letref}~ x \mathrel{=} v~ \kw{in}~ e} \tstep \\
   \tctx{\Sigma\left[x \mapsto \ell\right]}{H\left[\ell \mapsto v\right]}{E}{\kw{tick}}{e} \\
   \textrm{where $\ell$ fresh}}

\infax[E-Deref]
  {\tctx{\Sigma}{H}{E}{\kw{tick}}{!x} \tstep
   \tctx{\Sigma}{H}{E}{\kw{tick}}{v} \\
   \textrm{where $\ell = \Sigma[x]$, $H[\ell] = v$}}

\infax[E-Set]
  {\tctx{\Sigma}{H}{E}{\kw{tick}}{x \mathrel{:=} v} \tstep
   \tctx{\Sigma}{H\left[\ell \mapsto v\right]}{E}{\kw{tick}}{\kw{return}~ ()} \\
   \textrm{where $\ell = \Sigma[x]$}}

\infax[E-Bind]
  {\tctx{\Sigma}{H}{E}{\kw{tick}}{x \mathrel{\leftarrow} v;~ c} \tstep
   \tctx{\Sigma[x \mapsto v]}{H}{E}{\kw{tick}}{c}}

\infax[E-Seq]
  {\tctx{\Sigma}{H}{E}{\kw{tick}}{\kw{return}~ v;~ c} \tstep
   \tctx{\Sigma}{H}{E}{\kw{tick}}{c}}

\infax[E-TakeWait]
  {\tctx{\Sigma}{H}{E}{\kw{tick}}{\kw{take}} \tstep
   \tctx{\Sigma}{H}{E}{\kw{wait}}{\kw{take}}}

\infax[E-TakeConsume]
  {\tctx{\Sigma}{H}{E}{\kw{cons}~ v}{\kw{take}} \tstep
   \tctx{\Sigma}{H}{E}{\kw{tick}}{\kw{return}~ v}}

\infax[E-Emit]
  {\tctx{\Sigma}{H}{E}{\kw{tick}}{\kw{emit}~ v} \tstep
   \tctx{\Sigma}{H}{E}{\kw{yield}~v}{\kw{return}~ ()}}

\infax[E-Repeat]
  {\tctx{\Sigma}{H}{E}{\kw{tick}}{\kw{repeat}~ c} \tstep
   \tctx{\Sigma}{H}{E}{\kw{tick}}{c ;~ \kw{repeat}~ c}}
}
\caption{Computation thread evaluation relation}
\label{fig:eval:comp}
\end{figure*}

\begin{figure*}
\figurebox{
\judgment{Machine Evaluation}{\fbox{$\mach{\Phi_1}{\proc{\thread{\Sigma_1}{H_1}{e_1}{\delta_1}}{q_{i_1}}{q_{o_1}}, \ldots} \mstep \mach{\Phi_1'}{\proc{\thread{\Sigma_1'}{H_1'}{e_1'}{\delta_1'}}{q_{i_1}'}{q_{o_1}'}, \ldots}$}}
\infrule[P-Tick]
  {\tctx{\Sigma_1}{H_1}{E}{\delta_1}{e_1} \tstep
   \tctx{\Sigma_2}{H_2}{E}{\delta_2}{e_2}}
  {\mctx{\Phi}{\ldots \proc{\thread{\Sigma_1}{H_1}{E}{\delta_1}}{q_i}{q_o} \ldots }{e_1} \mstep
   \mctx{\Phi}{\ldots \proc{\thread{\Sigma_2}{H_2}{E}{\delta_2}}{q_i}{q_o} \ldots }{e_2}}

\infax[P-Consume]
  {\mctx{\Phi_1}{\ldots \proc{\thread{\Sigma}{H}{E}{\kw{wait}}}{q_i}{q_o} \ldots }{e} \mstep \\
   \mctx{\Phi_2}{\ldots \proc{\thread{\Sigma}{H}{E}{\kw{cons}~ v}}{q_i}{q_o} \ldots }{e} \\
   \text{where $(\Phi_2; v) = dequeue(\Phi_1; q_i)$}}

\infax[P-Yield]
  {\mctx{\Phi}{\ldots \proc{\thread{\Sigma}{H}{E}{\kw{yield}~ v}}{q_i}{q_o} \ldots }{e} \mstep \\
   \mctx{enqueue(\Phi; q_o; v)}{\ldots \proc{\thread{\Sigma}{H}{E}{\kw{tick}}}{q_i}{q_o} \ldots }{e}}

\infax[P-Spawn]
  {\mach{\Phi}{\ldots \proc{\thread{\Sigma}{H}{e_1 \arr e_2}{\kw{tick}}}{q_i}{q_o} \ldots } \mstep \\
   \mach{q = \kw{mvar}, \Phi}
   {\ldots \proc{\thread{\Sigma}{H}{e_1}{\kw{tick}}}{q_i}{q},
     \proc{\thread{\Sigma}{H}{e_1}{\kw{tick}}}{q}{q_o} \ldots} \\
   \text{where $q$ fresh}}

\infax[P-ParSpawn]
  {\mach{\Phi}{\ldots \proc{\thread{\Sigma}{H}{e_1 \arr e_2}{\kw{tick}}}{q_i}{q_o} \ldots } \mstep \\
   \mach{q = \kw{queue}, \Phi}
   {\ldots \proc{\thread{\Sigma_1}{H}{e_1}{\kw{tick}}}{q_i}{q},
     \proc{\thread{\Sigma_2}{H}{e_1}{\kw{tick}}}{q}{q_o} \ldots} \\
   \text{where $q$ fresh, $\Sigma_1 \oplus \Sigma_2 = \Sigma$}}
}
\caption{Machine evaluation relation}
\label{fig:eval:mach}
\end{figure*}

\begin{figure*}
\figurebox{
\begin{align*}
dequeue(\ldots, q = \kw{queue}~ v_1~ v_2~ \ldots, \ldots)
  &= (\ldots, q = \kw{queue}~ v_2~ \ldots, \ldots; v_1) \\
dequeue(\ldots, q = \kw{mvar}~ v, \ldots)
  &= (\ldots, q =\kw{mvar}, \ldots; v) \\ \\
enqueue(\ldots, q = \kw{queue}~ v_1 \ldots, \ldots; q; v)
  &= \ldots, q = \kw{queue}~ v_1 \ldots~ v, \ldots \\
enqueue(\ldots, q = \kw{mvar}~, \ldots; q; v)
  &= \ldots, q = \kw{mvar}~ v, \ldots
\end{align*}
}
\caption{Definition of the $dequeue$ and $enqueue$ functions}
\label{def:queue}
\end{figure*}

\end{document}
