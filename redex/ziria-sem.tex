% -*- coding: utf-8; mode: latex; mode: visual-line; mode: flyspell -*-
\documentclass[twocolumn]{article}

\usepackage{etex}

\usepackage[utf8]{inputenc}
\usepackage[T1]{fontenc}
\usepackage[english]{babel}
\usepackage[babel=true,activate={true,nocompatibility},final,tracking=true,kerning=true,spacing=true]{microtype}
\usepackage{lmodern}

\usepackage{amsmath}
\usepackage{amssymb}
\usepackage{bcprules}
\usepackage{multicol}
\usepackage{stmaryrd}
\usepackage{xspace}

\newcommand{\bnfdef}{::=}
\newcommand{\bnfalt}{\mathrel{\mid}}
\newcommand{\kw}[1]{\textsf{#1}}
\newcommand{\arr}{\mathrel{<\!\!<\!\!<}}
\newcommand{\types}[3]{#1 \vdash #2 \mathrel{:} #3}
\newcommand{\join}[2]{#1 \sqcup #2}
\newcommand{\clos}[3]{\left<#1; #2; #3\right>}

\newcommand{\thread}[5]{\left<#1; #2; #3; #4; #5\right>}
\newcommand{\proc}[3]{\ll\!#1; #2; #3\!\gg}
\newcommand{\mach}[2]{\parallel\!#1; #2\!\parallel}

\newcommand{\queue}[1]{\left<#1\right>}

\newcommand{\ox}[1]{\llbracket#1\rrbracket}
\newcommand{\tstep}{\rightarrow}
\newcommand{\mstep}{\Rightarrow}
\newcommand{\rref}[1]{\textsc{#1}}

\newcommand{\figurebox}[1]{\fbox{\begin{minipage}{\textwidth} #1 \medskip\end{minipage}}}

\newcommand{\judgment}[2]{{\bf #1} \hfill #2}

\typicallabel{P-ParSpawn}
\bcprulestwocoltrue
%%\bcprulessavespacetrue

\title{Ziria Operational Semantics}
\author{Geoffrey Mainland}

\begin{document}
\maketitle

\section{Language and Typing Relation}

The semantics given in this document are implemented in PLT Redex in the file \texttt{ziria.rkt}.

The expression language in Figure~\ref{fig:lang:exp} is essentially identical to the
existing Ziria language, but all syntactic categories have been collapsed into
the single category of expressions. In exchange, the type language in
Figure~\ref{fig:lang:type} is rather complicated.

Base types, $\tau$, are as expected. Computation types, represented by $\mu$,
are of the form $\kw{ST}~ \iota~ \alpha~ \beta$. Function arguments, of type
$\rho$, may be of type $\tau$ or $\kw{ref}~ \tau$. Function results, $\sigma$,
may be of type $\tau$ or $\mu$; that is, functions may return computations. The
typing context may contain values of the form $\rho$ (base types and \kw{ref}s)
or $\rho_1 \ldots \rho_n \to \sigma$ (functions). The typing judgment assigns
expressions a type of the form $\theta$, which may be a base type, \kw{ref}
type, computation, or function.

\begin{figure}
\begin{equation*}
\begin{array}{lcl}
  e, c
     & \bnfdef & k \\
     & \bnfalt & x \\
     & \bnfalt & unop~ e \\
     & \bnfalt & e~ binop~ e \\
     & \bnfalt & \kw{if}~ e~ \kw{then}~ e~ \kw{else}~ e \\
     & \bnfalt & \kw{let}~ x \mathrel{=} e~ \kw{in}~ e \\
     & \bnfalt & \kw{letfun}~ f~ (x_1 : \rho_1, \ldots, x_n : \rho_n) \mathrel{=} e~ \kw{in}~ e \\
     & \bnfalt & \kw{letref}~ x \mathrel{=} e~ \kw{in}~ e \\
     & \bnfalt & f~ e_1~ \ldots~ e_n \\
     & \bnfalt & \kw{!} x \\
     & \bnfalt & x \mathrel{:=} e \\
     & \bnfalt & \kw{return}~ e \\
     & \bnfalt & x \mathrel{\leftarrow} c;~ c \\
     & \bnfalt & c;~ c \\
     & \bnfalt & \kw{take} \\
     & \bnfalt & \kw{emit}~ e \\
     & \bnfalt & \kw{repeat}~ c \\
     & \bnfalt & c \arr c \\
  k
     & \bnfdef & () \\
     & \bnfalt & \kw{true} \\
     & \bnfalt & \kw{false} \\
     & \bnfalt & n \\
  unop
     & \bnfdef & - \\
  binop
     & \bnfdef & \ne \\
     & \bnfalt & = \\
     & \bnfalt & + \\
     & \bnfalt & - \\
     & \bnfalt & \times \\
     & \bnfalt & \kw{div}
\end{array}
\end{equation*}
\caption{Expression language}
\label{fig:lang:exp}
\end{figure}

\begin{figure}
\begin{equation*}
\begin{array}{lcl}
  \alpha, \beta, \gamma, \nu, \tau
     & \bnfdef & () \\
     & \bnfalt & \kw{bool} \\
     & \bnfalt & \kw{int} \\
  \iota
     & \bnfdef & \kw{C}~ \tau \\
     & \bnfalt & \kw{T} \\
  \mu
     & \bnfdef & \kw{ST}~ \iota~ \tau~ \tau~ \\
  \rho
     & \bnfdef & \tau \\
     & \bnfalt & \kw{ref}~ \tau \\
  \sigma
     & \bnfdef & \tau \\
     & \bnfalt & \mu \\
  \phi
     & \bnfdef & \rho \\
     & \bnfalt & \rho_1 \ldots \rho_n \to \sigma \\
  \Gamma
     & \bnfdef & \cdot \\
     & \bnfalt & x : \phi, \Gamma \\
  \theta
     & \bnfdef & \tau \\
     & \bnfalt & \kw{ref}~ \tau \\
     & \bnfalt & \mu \\
     & \bnfalt & \rho_1 \ldots \rho_n \to \sigma
\end{array}
\end{equation*}
\caption{Type language}
\label{fig:lang:type}
\end{figure}

The base typing rules are given in Figure~\ref{type:base}. The rules for
computations are given in Figure~\ref{type:comp}. Both monadic sequencing and
monadic bind are included because treating sequencing as a derived form was
slightly more complicated in Redex---it was easier not to generate a fresh
variable for the binding. Notice that $\kw{map}$ has been removed.

\subsection{Context splitting and heap splitting}

The context splitting operation $\oplus$ only splits the portion of the context
that contains variables that have type $\kw{ref}~ \tau$; the rest of the context
may be shared. The reduction semantics \emph{does not} split the heap; a
well-typed program will not share \kw{ref} values between different threads.

\begin{figure*}
\figurebox{
\begin{multicols}{2}[\judgment{Typing Relation}{\fbox{$\types{\Gamma}{e}{\theta}$}}]
\infax[T-Unit]
  {\types{}{()}{()}}

\infax[T-True]
  {\types{}{\kw{true}}{\kw{bool}}}

\infax[T-False]
  {\types{}{\kw{false}}{\kw{bool}}}

\infax[T-Int]
  {\types{}{n}{\kw{bool}}}

\infrule[T-Var]
  {x:\phi \in \Gamma}
  {\types{\Gamma}{x}{\phi}}

\infrule[T-Unop]
  {\types{\Gamma}{e}{\phi_1}}
  {\types{\Gamma}{unop~ e}{unop_{\phi_1}(unop)}}

\infrule[T-Binop]
  {\types{\Gamma}{e_1}{\phi_1} \andalso \types{\Gamma}{e_2}{\phi_2}}
  {\types{\Gamma}{e_1~ binop~ e_2}{binop_{\phi_1, \phi_2}(binop)}}

\infrule[T-If]
  {\types{\Gamma}{e_1}{\kw{bool}} \andalso
    \types{\Gamma}{e_2}{\phi} \andalso
    \types{\Gamma}{e_3}{\phi}}
  {\types{\Gamma}{\kw{if}~ e_1~ \kw{then}~ e_2~ \kw{else}~ e_3}{\phi}}

\infrule[T-Let]
  {\types{\Gamma}{e_1}{\tau_1} \andalso
    \types{x : \tau_1, \Gamma}{e_2}{\phi}}
  {\types{\Gamma}{\kw{let}~ x \mathrel{=} e_1~ \kw{in}~ e_2}{\phi}}

\infrule[T-LetFun]
  {\types{x_1 : \rho_1, \ldots, x_n : \rho_n, \Gamma}{e_1}{\sigma} \\
    \types{f : \rho_1 \ldots \rho_n \to \sigma, \Gamma}{e_2}{\theta}}
  {\types{\Gamma}{\kw{letfun}~ f~ (x_1 : \rho_1, \ldots, x_n : \rho_n) \mathrel{=} e_1~ \kw{in}~ e_2}{\phi}}

\infrule[T-Call]
  {\types{\Gamma}{f}{\rho_1 \ldots \rho_n \to \sigma} \\
    \types{\Gamma}{e_1}{\rho_1} \andalso \ldots \andalso \types{\Gamma}{e_n}{\rho_n}}
  {\types{\Gamma}{f~ e_1~ \ldots e_n}{\sigma}}
\end{multicols}
}
\caption{Base typing rules}
\label{type:base}
\end{figure*}

\begin{figure*}
\figurebox{
\begin{multicols}{2}[\judgment{Typing Relation}{\fbox{$\types{\Gamma}{e}{\theta}$}}]
\infrule[T-LetRef]
  {\types{\Gamma}{e_1}{\tau_1} \andalso
    \types{x : \kw{ref}~ \tau_1, \Gamma}{e_2}{\kw{ST}~ \iota~ \alpha~ \beta}}
  {\types{\Gamma}{\kw{letref}~ x \mathrel{=} e_1~ \kw{in}~ e_2}{\kw{ST}~ \iota~ \alpha~ \beta}}

\infrule[T-Deref]
  {x:\kw{ref}~ \tau \in \Gamma}
  {\types{\Gamma}{!x}{\kw{ST}~ (\kw{C}~ \tau)~ \alpha~ \beta}}

\infrule[T-Set]
  {x:\kw{ref}~ \tau \in \Gamma \andalso \types{\Gamma}{e}{\tau}}
  {\types{\Gamma}{x \mathrel{:=} e}{\kw{ST}~ (\kw{C}~ ())~ \alpha~ \beta}}

\infrule[T-Return]
  {\types{\Gamma}{c}{\tau}}
  {\types{\Gamma}{\kw{return}~ c}{\kw{ST}~ (\kw{C}~ \tau)~ \alpha~ \beta}}

\infrule[T-Bind]
  {\types{\Gamma}{c_1}{\kw{ST}~ (\kw{C}~ \nu)~ \alpha~ \beta} \andalso
   \types{x : \nu, \Gamma}{c_2}{\kw{ST}~ \iota~ \alpha~ \beta}}
  {\types{\Gamma}{x \mathrel{\leftarrow} c_1;~ c_2}{\kw{ST}~ \iota~ \alpha~ \beta}}

\infrule[T-Seq]
  {\types{\Gamma}{c_1}{\kw{ST}~ (\kw{C}~ \nu)~ \alpha~ \beta} \andalso
   \types{x\Gamma}{c_2}{\kw{ST}~ \iota~ \alpha~ \beta}}
  {\types{\Gamma}{c_1;~ c_2}{\kw{ST}~ \iota~ \alpha~ \beta}}

\infax[T-Take]
  {\types{\Gamma}{\kw{take}}{\kw{ST}~ (\kw{C}~ \alpha)~ \alpha~ \beta}}

\infrule[T-Emit]
  {\types{\Gamma}{e}{\beta}}
  {\types{\Gamma}{\kw{emit}~ e}{\kw{ST}~ (\kw{C}~ ())~ \alpha~ \beta}}

\infrule[T-Repeat]
  {\types{\Gamma}{c}{\kw{ST}~ (\kw{C}~ ())~ \alpha~ \beta}}
  {\types{\Gamma}{\kw{repeat}~ c}{\kw{ST}~ \kw{T}~ \alpha~ \beta}}

\infrule[T-Arr]
  {\Gamma_1 \oplus \Gamma_2 = \Gamma \\
   \types{\Gamma_1}{c_1}{\kw{ST}~ \iota_1~ \alpha~ \beta} \\
   \types{\Gamma_2}{c_2}{\kw{ST}~ \iota_2~ \alpha~ \beta}}
  {\types{\Gamma}{c_1 \arr c_2}{\kw{ST}~ (\join{\iota_1}{\iota_2})~ \alpha~ \beta}}
\end{multicols}
}
\caption{Computation typing rules}
\label{type:comp}
\end{figure*}

\begin{figure}
\begin{align*}
\join{\kw{C}~ \nu}{\kw{T}} &= \kw{C}~ \nu \\
\join{\kw{T}}{\kw{C}~ \nu} &= \kw{C}~ \nu \\
\join{\kw{T}}{\kw{T}}      &= \kw{T}
\end{align*}
\caption{Definition of $\join{\cdot}{\cdot}$}
\label{def:join}
\end{figure}

\section{Evaluation}

\begin{figure}
\begin{equation*}
\begin{array}{lcl}
  \kappa
     & \bnfdef & \kw{halt} \\
     & \bnfalt & \kw{popk}~ \Sigma~ \kappa \\
     & \bnfalt & \kw{unopk}~ unop~ \kappa \\
     & \bnfalt & \kw{binop1k}~ binop~ e~ \kappa \\
     & \bnfalt & \kw{binop2k}~ binop~ v~ \kappa  \\
     & \bnfalt & \kw{ifk}~ e~ e~ \kappa \\
     & \bnfalt & \kw{letk}~ x~ e~ \kappa \\
     & \bnfalt & \kw{argk}~ f~ (v_1, \ldots, v_{i-1})~ (e_{i+1}, \ldots e_n)~ \kappa \\
     & \bnfalt & \kw{letrefk}~ x~ e~ \kappa \\
     & \bnfalt & \kw{setk}~ x~ \kappa \\
     & \bnfalt & \kw{returnk}~ \kappa \\
     & \bnfalt & \kw{bindk}~ x~ e~ \kappa \\
     & \bnfalt & \kw{seqk}~ e~ \kappa \\
     & \bnfalt & \kw{emitk} \\
  v
     & \bnfdef & k \\
  clos
     & \bnfdef & \clos{x_1, \ldots, x_n}{e}{\Sigma} \\
  \Sigma
     & \bnfdef & \emptyset \\
     & \bnfalt & x = \ell, \Sigma \\
  v_H
     & \bnfdef & v \\
     & \bnfalt & clos \\
  H
     & \bnfdef & \emptyset \\
     & \bnfalt & \ell = v_H, H \\
  \delta
     & \bnfdef & \kw{tick} \\
     & \bnfalt & \kw{wait} \\
     & \bnfalt & \kw{cons}~ v \\
     & \bnfalt & \kw{yield}~ v \\
  t
     & \bnfdef & \thread{\Sigma}{H}{e}{\kappa}{\delta} \\
  Q
     & \bnfdef & \kw{queue}~ v \ldots \\
  p
     & \bnfdef & \proc{\thread{\Sigma}{H}{e}{\kappa}{\delta}}{q_i}{q_o} \\
  \Phi
     & \bnfdef & \emptyset \\
     & \bnfalt & q = Q, \Phi \\
\end{array}
\end{equation*}
\caption{Evaluation contexts}
\label{fig:lang:context}
\end{figure}

The evaluation relation is formulated as a CESK-style abstract machine, shown in
Figure~\ref{fig:lang:context}. Closures are represented by $clos$. All values
are heap-allocated; the store, $\Sigma$, maps variables to locations, and the
heap, $H$, maps locations to values or closures.

\emph{Threads} $t$ consist of a store, a heap, an expression, a continuation
$\kappa$, and an \emph{action} $\delta$. This action represents the thread's
interaction with the outside world via queues. \emph{Processes} $p$ consist of a
thread and two queues, an input queue and an output queue. \emph{Machines}
$\mach{\Phi}{p_1,\ldots,p_n}$ consist of a ``queue heap,'' $\Phi$, that maps
queue locations $q$ to queues, and a collection of processes. Queues can contain
zero or more values.

Evaluation is specified in the form of two small-step relations, one for
threads, and one for machines. Figure~\ref{fig:eval:base} and
Figure~\ref{fig:eval:comp} give the thread evaluation relation. The $\kw{take}$
and $\kw{emit}$ expressions cause the thread to synchronize via the $\delta$
action; $\kw{take}$ waits for a value, and $\kw{emit}$ yields a value. The nice
thing about this alternative (with respect to the semantics from the NSDI
submission) formulation is that is is small-step, allowing us to reason about
threads running in parallel, and the domain and codomain are identical.

The evaluation relation for machines is given in Figure~\ref{fig:eval:mach}. If
one thread can step, then the machine can step. If a thread needs to synchronize
via its input/output queues, then the machine can step by either enqueuing or
dequeuing values to satisfy a thread's action, $\delta$.

\begin{figure*}
\figurebox{
\judgment{Thread Evaluation}
{\fbox{$\thread{\Sigma_1}{H_1}{e_1}{\kappa_1}{\delta_1} \tstep \thread{\Sigma_2}{H_2}{e_2}{\kappa_2}{\delta_2}$}}
\infax[E-PopK]
  {\thread{\Sigma}{H}{v}{\kw{popk}~ \Sigma'~ \kappa}{\kw{tick}} \tstep
   \thread{\Sigma'}{H}{v}{\kappa}{\kw{tick}}}

\infax[E-PopReturnK]
  {\thread{\Sigma}{H}{\kw{return}~v}{\kw{popk}~ \Sigma'~ \kappa}{\kw{tick}} \tstep
   \thread{\Sigma'}{H}{\kw{return}~v}{\kappa}{\kw{tick}}}

\infax[E-Var]
  {\thread{\Sigma}{H}{x}{\kappa}{\kw{tick}} \tstep
   \thread{\Sigma}{H}{v}{\kappa}{\kw{tick}}
   \quad \mbox{where $\ell = \Sigma\left[x\right], v = H\left[\ell\right]$}}

\infax[E-Unop]
  {\thread{\Sigma}{H}{unop~ e}{\kappa}{\kw{tick}} \tstep
   \thread{\Sigma}{H}{e}{\kw{unopk}~ unop~ \kappa}{\kw{tick}}}

\infax[E-UnopK]
  {\thread{\Sigma}{H}{v}{\kw{unopk}~unop~ \kappa}{\kw{tick}} \tstep
    \thread{\Sigma}{H}{\ox{unop~v}}{\kappa}{\kw{tick}}}

\infax[E-Binop]
  {\thread{\Sigma}{H}{binop~ e_1~ e_2}{\kappa}{\kw{tick}} \tstep
   \thread{\Sigma}{H}{e_1}{\kw{binopk1}~ binop~ e_2~ \kappa}{\kw{tick}}}

\infax[E-Binop1K]
  {\thread{\Sigma}{H}{v_1}{\kw{binop1k}~ binop~ e_2~ \kappa}{\kw{tick}} \tstep
    \thread{\Sigma}{H}{e_2}{\kw{binop2k}~ binop~ v_1~ \kappa}{\kw{tick}}}

\infax[E-Binop2K]
  {\thread{\Sigma}{H}{v_2}{\kw{binop2k}~ binop~ v_1~ \kappa}{\kw{tick}} \tstep
    \thread{\Sigma}{H}{\ox{v_1~ binop~ v_2}}{\kappa}{\kw{tick}}}

\infax[E-If]
  {\thread{\Sigma}{H}{\kw{if}~ e_1~ \kw{then}~ e_2~ \kw{else}~ e_3}{\kappa}{\kw{tick}} \tstep
   \thread{\Sigma}{H}{e_1}{\kw{ifk}~ e_2~ e_3~ \kappa}{\kw{tick}}}

\infax[E-IfTrueK]
  {\thread{\Sigma}{H}{\kw{true}}{\kw{ifk}~ e_1~ e_2~ \kappa}{\kw{tick}} \tstep
   \thread{\Sigma}{H}{e_1}{\kappa}{\kw{tick}}}

\infax[E-IfFalseK]
  {\thread{\Sigma}{H}{\kw{false}}{\kw{ifk}~ e_1~ e_2~ \kappa}{\kw{tick}} \tstep
   \thread{\Sigma}{H}{e_2}{\kappa}{\kw{tick}}}

\infax[E-Let]
  {\thread{\Sigma}{H}{\kw{let}~ x = e_1~ \kw{in}~ e_2}{\kappa}{\kw{tick}}\tstep
   \thread{\Sigma}{H}{e_1}{\kw{letk}~ x~ e_2~ \kappa}{\kw{tick}}}

\infax[E-LetK]
  {\thread{\Sigma_1}{H_1}{v}{\kw{letk}~ x~ e_2~ \kappa}{\kw{tick}}\tstep
   \thread{\Sigma_2}{H_2}{e_2}{\kw{popk}~ \Sigma_1~ \kappa}{\kw{tick}} \\
   \quad \mbox{where $\ell$ fresh, $\Sigma_2 = \Sigma_1\left[x \mapsto \ell\right], H_2 = H_1\left[\ell \mapsto v\right]$}}

\infax[E-LetFun]
  {\thread{\Sigma_1}{H_1}{\kw{letfun}~ f~ (x_1 : \rho_1, \ldots, x_n : \rho_n) \mathrel{=} e_1~ \kw{in}~ e_2}{\kappa}{\kw{tick}}\tstep
  \thread{\Sigma_2}{H_2}{e_2}{\kw{popk}~ \Sigma_1~ \kappa}{\kw{tick}} \\
  \quad \mbox{where $\ell$ fresh, $\Sigma_2 = \Sigma_1\left[f \mapsto \ell\right], H_2 = H_1\left[\ell \mapsto \clos{x_1, \ldots, x_n}{e_1}{\Sigma_2}\right]$}}

\infax[E-Call]
  {\thread{\Sigma}{H}{v_i}{\kw{argk}~ f~ (v_1, \ldots, v_{i-1})~ (e_2, \ldots, e_n)~ \kappa}{\kw{tick}} \tstep
    \thread{\Sigma}{H}{e_1}{\kw{argk}~ f~ ()~ (e_2, \ldots, e_n)~ \kappa}{\kw{tick}}}

\infax[E-ArgK]
  {\thread{\Sigma}{H}{v_i}{\kw{argk}~ f~ (v_1, \ldots, v_{i-1})~ (e_{i+1}, \ldots e_n)~ \kappa}{\kw{tick}} \tstep \\
   \thread{\Sigma}{H}{e_{i+1}}{\kw{argk}~ f~ (v_1, \ldots, v_{i-1}, v_i)~ (e_{i+2}, \ldots e_n)~ \kappa}{\kw{tick}}}

\infax[E-CallK]
  {\thread{\Sigma_1}{H_1}{v_n}{\kw{argk}~ f~ (v_1, \ldots, v_{n-1})~ ()~ \kappa}{\kw{tick}} \tstep \\
   \thread{\Sigma_2}{H_2}{e_f}{\kw{popk}~ \Sigma_1~ \kappa}{\kw{tick}} \\
     \mbox{where $\ell_f = \Sigma_1[f], \clos{x_1, \ldots, x_n}{e_f}{\Sigma_f} = H_1[\ell_f]$} \\
     \mbox{$\ell_1, \ldots, \ell_n$ fresh,} \\
     \mbox{$\Sigma_2 = \Sigma_1\left[ \overline{x_i \mapsto \ell_i} \right]$} \\
     \mbox{$H_2      = H_1\left[ \overline{\ell_i \mapsto v_i} \right]$}
   }
}
\caption{Base thread evaluation relation}
\label{fig:eval:base}
\end{figure*}

\begin{figure*}
\figurebox{
{\fbox{$\thread{\Sigma_1}{H_1}{e_1}{\kappa_1}{\delta_1} \tstep \thread{\Sigma_2}{H_2}{e_2}{\kappa_2}{\delta_2}$}}

\infax[E-LetRef]
  {\thread{\Sigma}{H}{\kw{letref}~ x = e_1~ \kw{in}~ e_2}{\kappa}{\kw{tick}}\tstep
   \thread{\Sigma}{H}{e_1}{\kw{letrefk}~ x~ e_2~ \kappa}{\kw{tick}}}

\infax[E-LetRefK]
  {\thread{\Sigma_1}{H_1}{v}{\kw{letrefk}~ x~ e_2~ \kappa}{\kw{tick}}\tstep
   \thread{\Sigma_2}{H_2}{e_2}{\kw{popk}~ \Sigma_1~ \kappa}{\kw{tick}} \\
     \quad \mbox{where $\ell$ fresh, $\Sigma_2 = \Sigma_1\left[x \mapsto \ell\right], H_2 = H_1\left[\ell \mapsto v\right]$}}

\infax[E-Deref]
  {\thread{\Sigma}{H}{!x}{\kappa}{\kw{tick}} \tstep
   \thread{\Sigma}{H}{\kw{return}~ v}{\kappa}{\kw{tick}} \\
   \textrm{where $\ell = \Sigma[x]$, $v = H[\ell]$}}

\infax[E-Set]
  {\thread{\Sigma}{H}{x \mathrel{:=} e}{\kappa}{\kw{tick}} \tstep
   \thread{\Sigma}{H}{e}{\kw{setk}~ x~ \kappa}{\kw{tick}}}

\infax[E-SetK]
  {\thread{\Sigma}{H_1}{v}{\kw{setk}~ x~ \kappa}{\kw{tick}} \tstep
   \thread{\Sigma}{H_2}{\kw{return}~ ()}{\kappa}{\kw{tick}} \\
   \textrm{where $\ell = \Sigma\left[ x \right], H_2 = H_1\left[ \ell \mapsto v \right]$}}

\infax[E-Return]
  {\thread{\Sigma}{H}{\kw{return}~ e}{\kappa}{\kw{tick}} \tstep
   \thread{\Sigma}{H}{e}{\kw{returnk}~ \kappa}{\kw{tick}} \\
   \textrm{where $e$ is not a value}}

\infax[E-ReturnK]
  {\thread{\Sigma}{H}{v}{\kw{returnk}~ \kappa}{\kw{tick}} \tstep
   \thread{\Sigma}{H}{\kw{return}~ v}{\kappa}{\kw{tick}}}

\infax[E-Bind]
  {\thread{\Sigma}{H}{x \mathrel{\leftarrow} e;~ c}{\kappa}{\kw{tick}} \tstep
   \thread{\Sigma}{H}{e}{\kw{bindk}~ x~ c~ \kappa}{\kw{tick}}}

\infax[E-BindK]
  {\thread{\Sigma_1}{H_1}{v}{\kw{bindk}~ x~ c~ \kappa}{\kw{tick}} \tstep
   \thread{\Sigma_2}{H_2}{c}{\kappa}{\kw{tick}} \\
   \quad \mbox{where $\ell$ fresh, $\Sigma_2 = \Sigma_1\left[x \mapsto \ell\right], H_2 = H_1\left[\ell \mapsto v\right]$}}

\infax[E-Seq]
  {\thread{\Sigma}{H}{c_1;~ c_2}{\kappa}{\kw{tick}} \tstep
   \thread{\Sigma}{H}{c_1}{\kw{seqk}~ c_2~ \kappa}{\kw{tick}}}

\infax[E-SeqK]
  {\thread{\Sigma}{H}{\kw{return}~ v}{\kw{seqk}~ c~ \kappa}{\kw{tick}} \tstep
   \thread{\Sigma}{H}{c}{\kappa}{\kw{tick}}}

\infax[E-TakeWait]
  {\thread{\Sigma}{H}{\kw{take}}{\kappa}{\kw{tick}} \tstep
   \thread{\Sigma}{H}{\kw{take}}{\kappa}{\kw{wait}}}

\infax[E-TakeConsume]
  {\thread{\Sigma}{H}{\kw{take}}{\kappa}{\kw{cons}~ v} \tstep
   \thread{\Sigma}{H}{\kw{return}~ v}{\kappa}{\kw{tick}}}

\infax[E-Emit]
  {\thread{\Sigma}{H}{\kw{emit}~ e}{\kappa}{\kw{tick}} \tstep
   \thread{\Sigma}{H}{e}{\kw{emitk}~ \kappa}{\kw{tick}}}

\infax[E-EmitK]
  {\thread{\Sigma}{H}{v}{\kw{emitk}~ \kappa}{\kw{tick}} \tstep
   \thread{\Sigma}{H}{\kw{return}~ ()}{\kappa}{\kw{yield}~ v}}

\infax[E-Repeat]
  {\thread{\Sigma}{H}{\kw{repeat}~ c}{\kappa}{\kw{tick}} \tstep
   \thread{\Sigma}{H}{c ;~ \kw{repeat}~ c}{\kappa}{\kw{tick}}}
}
\caption{Computation thread evaluation relation}
\label{fig:eval:comp}
\end{figure*}

\begin{figure*}
\figurebox{
\judgment{Machine Evaluation}{\fbox{$\mach{\Phi_1}{\proc{\thread{\Sigma_1}{H_1}{e_1}{\delta_1}}{q_{i_1}}{q_{o_1}}, \ldots} \mstep \mach{\Phi_1'}{\proc{\thread{\Sigma_1'}{H_1'}{e_1'}{\delta_1'}}{q_{i_1}'}{q_{o_1}'}, \ldots}$}}
\infrule[P-Tick]
  {\thread{\Sigma_1}{H_1}{e_1}{\kappa_1}{\delta_1} \tstep
   \thread{\Sigma_2}{H_2}{e_2}{\kappa_2}{\delta_2}}
  {\mach{\Phi}{\ldots \proc{\thread{\Sigma_1}{H_1}{e_1}{\kappa_1}{\delta_1}}{q_i}{q_o} \ldots } \mstep
   \mach{\Phi}{\ldots \proc{\thread{\Sigma_2}{H_2}{e_2}{\kappa_2}{\delta_2}}{q_i}{q_o} \ldots }}

\infax[P-Consume]
  {\mach{\Phi_1}{\ldots \proc{\thread{\Sigma}{H}{e}{\kappa}{\kw{wait}}}{q_i}{q_o} \ldots } \mstep \\
   \mach{\Phi_2}{\ldots \proc{\thread{\Sigma}{H}{e}{\kappa}{\kw{cons}~ v}}{q_i}{q_o} \ldots } \\
   \text{where $(\Phi_2; v) = dequeue(\Phi_1; q_i)$}}

\infax[P-Yield]
  {\mach{\Phi}{\ldots \proc{\thread{\Sigma}{H}{e}{\kappa}{\kw{yield}~ v}}{q_i}{q_o} \ldots } \mstep \\
   \mach{enqueue(\Phi; q_o; v)}{\ldots \proc{\thread{\Sigma}{H}{e}{\kappa}{\kw{tick}}}{q_i}{q_o} \ldots }}

\infax[P-Spawn]
  {\mach{\Phi}{\ldots \proc{\thread{\Sigma}{H}{e_1 \arr e_2}{\kappa}{\kw{tick}}}{q_i}{q_o} \ldots } \mstep \\
   \mach{q = \kw{queue}, \Phi}
   {\ldots \proc{\thread{\Sigma}{H}{e_1}{\kappa}{\kw{tick}}}{q_i}{q},
     \proc{\thread{\Sigma}{H}{e_2}{\kappa}{\kw{tick}}}{q}{q_o} \ldots} \\
   \text{where $q$ fresh}}
}
\caption{Machine evaluation relation}
\label{fig:eval:mach}
\end{figure*}

\begin{figure*}
\figurebox{
\begin{align*}
dequeue(\ldots, q = \kw{queue}~ v_1~ v_2~ \ldots, \ldots)
  &= (\ldots, q = \kw{queue}~ v_2~ \ldots, \ldots; v_1) \\
enqueue(\ldots, q = \kw{queue}~ v_1 \ldots, \ldots; q; v)
  &= \ldots, q = \kw{queue}~ v_1 \ldots~ v, \ldots
\end{align*}
}
\caption{Definition of the $dequeue$ and $enqueue$ functions}
\label{def:queue}
\end{figure*}

\end{document}
